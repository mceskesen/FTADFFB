\chapter{Summary (Danish)}
\begin{otherlanguage}{danish}

Mikrofluidiske biochips revolutionerer biologi ved at lægge laboratorie funktionalitet på en meget lille chip. Denne afhandling fokuserer på flow-baserede biochips. Flow-baserede mikrofluidiske biochips er baseret på behandling af kontinuerlig væske gennem fabrikeret mikrokanaler ved at bruge eksterne trykkilder eller integreret mikro-pumper. I disse biochips er byggestenen en mikrovalve som kan blive fabrikeret ved høj tæthed, fx. 1 million valves per cm$^2$. Ved at kombinere disse valves kan mere komplekse enheder fremstilles - herunder mixere, switches, multipleksere. Flow-baserede biochips er fremstillet ved brug af multilags blød litografi.

En potentiel forhindring i implementeringen af mikrofluidiske biochips er manglen på testteknikker til at screene defekte enheder før de bruges til biokemisk analyse. Defekte chips fører til gentagelse af eksperimenter. Dette er uhensigtsmæssigt på grund af høje reagens omkostninger og begrænset tilgængelighed af prøver. Flow-baserede biochips er også påvirket af fejl, og disse defekter kan undslippe efter-fabrikation inspektion og dermed påvirke operationen. Fejlmodellering og automatiske test af flow-baserede biochips er blevet adresseret fornyligt i en teknisk rapport.

Baseret på disse fejlmodeller og fejlfindingsteknikker er målet for denne afhandling at foreslå tilgange til det fejltolerante design af flow-baserede biochips således at biochippen kan tolerere flere permanente fejl givet et budget og et biochip område. Under det fysiske design af biochippens layout kan redundans introduceres for komponenterne på chippen såsom valves, kanaler og microfluidiske enheder, og derved øge udbyttet af biochips.

Denne afhandling foreslår en fejlmodel som en del af biochip arkitekturmodellen og introducerer et komponentbibliotek med fejltolerante komponenter. Ved brug af disse modeller er to algoritmiske tilgange foreslået til at løse problemet med fremstilling af fejltolerante arkitekturer. Fremstillingen af fejltolerante arkitektur tager applikationsmodellen og fejltolerant rutebestemmelse på biochippen med i overvejelserne samt de fysiske begrænsinger således at den minimale mængde af redundans er tilføjet for at opnå fejltolerance. De foreslåede tilgange er blevet evalueret ved brug af real-life case studies og syntetiske benchmarks.

\end{otherlanguage}