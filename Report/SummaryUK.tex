\chapter{Summary (English)}

Microfluidic biochips revolutionise biology by placing laboratory functionality on a very small chip. In this thesis, the focus is on flow-based biochips. Flow-based microfluidic biochips are used for the manipulation of continuous fluid through fabricated microchannels, using external pressure sources or integrated mechanical micro-pumps. In these biochips, the basic building block is a microvalve, which can be fabricated at very high densities, e.g., 1 million valves per cm$^2$ \cite{wajid}. By combining these valves, more complex units such as mixers, switches and multiplexers can be built. Flow-based biochips are manufactured using multilayer soft lithography.

A potential roadblock in the deployment of microfluidic biochips is the lack of test techniques to screen defective devices before they are used for biochemical analysis. Defective chips lead to repetition of experiments. This is undesirable due to high reagent cost and limited availability of samples. Flow-based biochips are also affected by faults, and the defects can escape the after-fabrication inspection and can thereby affect the operation. Recent work has addressed fault-modeling and the automated testing of flow-based biochips.

Based on these fault models and testing techniques, the objective of this thesis is to propose approaches for the fault-tolerant design of flow-based biochips, such that the biochips can tolerate several permanent faults, given a cost budget and a biochip area. During the physical design of the biochip layout, redundancy can be introduced for on-chip components such as valves, channels and microfluidic units in order to improve fault-tolerance, thereby increasing thus the yield.

The thesis proposes a fault model as part of the biochip architecture model and introduces a component library with fault-tolerant components. Using these models, two algorithmic approaches to solving the problem of fault-tolerant architecture synthesis are proposed. The fault-tolerant architecture synthesis considers the application model, fault-tolerant routing and the physical constraints of the biochip such that the minimum amount of redundancy is added to achieve fault-tolerance. The proposed approaches have been evaluated using real-life case studies and synthetic benchmarks.